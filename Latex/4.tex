%!TEX program=xelatex
\documentclass[UTF8]{ctexart}
\usepackage{amsmath}        
\usepackage{geometry}
\geometry{left=2.5cm, right=2.5cm, top=2.5cm, bottom=2.5cm}  %重设页边距(默认的页边距太大,不美观)
\usepackage{xeCJK}      %字体包
%\usepackage{fontspec}
\begin{document}

旋转变换则要复杂许多,一个比较好的旋转模型是,绕一个给定的轴$\vec n$,
围绕$\vec n$旋转$\theta$角。下面给出旋转变换的原理和公式推导方法:
假设给出的转轴$\vec n$长度为1(即$\vec n$是转轴方向上的单位向量),
要旋转的向量(点也成立,这里以向量为例)为$\vec v$,
完成旋转后的向量为$R_n(\vec v)$。

首先将$\vec v$分解为两部分,一部分平行于
$\vec n$而另一部分垂直于
$\vec n$,
平行于$\vec n$的分量即为
$\vec v$在$\vec n$上的投影
$Proj_n(\vec v)$,
则另一分量可以得出$\vec{v_{\perp}} = \vec{v} - Proj_n(\vec v)$,
接下来求得在图中与轴$\vec n$垂直的圆平面上,$\vec{v_{\perp}}$旋转$\theta$后的$R_n(\vec{v_{\perp}})$,为此我们在该
圆平面上找两个正交的向量作为基,一个向量就是$\vec{v_{\perp}}$,而另一个向量则取$\vec n \times \vec v$($\times$是向量叉乘)




\end{document}