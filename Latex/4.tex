%!TEX program=xelatex
\documentclass[UTF8]{ctexart}
\usepackage{amsmath}        
\usepackage{geometry}
\geometry{left=2.5cm, right=2.5cm, top=2.5cm, bottom=2.5cm}  %重设页边距(默认的页边距太大,不美观)
\usepackage{xeCJK}      %字体包
%\usepackage{fontspec}
\begin{document}

旋转变换则要复杂许多,一个比较好的旋转模型是,绕一个给定的轴$\vec n$,
围绕$\vec n$旋转$\theta$角。下面给出旋转变换的原理和公式推导方法:
假设给出的转轴$\vec n$长度为1(即$\vec n$是转轴方向上的单位向量),
要旋转的向量(点也成立,这里以向量为例)为$\vec v$,
完成旋转后的向量为$R_n(\vec v)$。

首先将$\vec v$分解为两部分,一部分平行于
$\vec n$而另一部分垂直于
$\vec n$,
平行于$\vec n$的分量即为
$\vec v$在$\vec n$上的投影
$Proj_n(\vec v)$,
则另一分量可以得出$\vec{v_{\perp}} = \vec{v} - Proj_n(\vec v)$,
接下来求得在图中与轴$\vec n$垂直的圆平面上,$\vec{v_{\perp}}$旋转$\theta$后的$R_n(\vec{v_{\perp}})$,为此我们在该
圆平面上找两个正交的向量作为基,一个向量就是$\vec{v_{\perp}}$,而另一个向量则取$\vec n \times \vec v$($\times$是向量叉乘):

这样就可以求得$R_n(\vec{v_{\perp}}) = \vec{v_{\perp}} \cos\theta + (\vec n \times \vec v) \sin\theta$,
那么最后要求的$R_n(\vec v) = R_n(\vec{v_{\perp}}) + Proj_n(\vec v)$,展开并做变形:

\begin{align}
R_n(\vec v)  &= R_n(\vec{v_{\perp}}) + Proj_n(\vec v) \\
    &= \vec{v_{\perp}}\cos\theta + (\vec n \times \vec v) \sin\theta + (\vec n \cdot \vec v)\vec n \\
    &= (\vec n \cdot \vec v)\vec n + (\vec v - (\vec n \cdot \vec v)\vec n)\cos\theta + (\vec n \times \vec v)\sin\theta \\
    &= \vec{v} \cos\theta + (\vec n \cdot \vec v)\vec{n}(1-\cos\theta) + (\vec n \times \vec v)\sin \theta 
\end{align}

依照上面的结果给出绕轴$n$旋转$\theta$角度的变换矩阵$R$,要求$\vec n$是单位向量,式中$x$, $y$, $z$分别是$\vec n$的三个分量,$c = \cos\theta$, $s = \sin\theta$

$R = \begin{bmatrix}
    c+(1-c)x^2 & (1-c)xy+sz & (1-c)xz-sy & 0 \\ 
    (1-c)xy-sz & c+(1-c)y^2 & (1-c)yz+sx & 0\\ 
    (1-c)xz+sy & (1-c)yz-sx & c+(1-c)z^2 & 0\\ 
    0 & 0 & 0 & 1 
    \end{bmatrix} $

值得注意的是,旋转变换的逆变换即反方向旋转$\theta$的变换,将$\theta$代换为$-\theta$后可以得到逆变换矩阵$R^{-1}$,它恰好等于$R$的转置$R^T$。即旋转变换矩阵$R$是一个正交矩阵。

\newpage

而平移变换就简单多了,我们想将坐标$(x, y, z, 1)$变换到$(x+T_x, y+T_y, z+T_z, 1)$,容易验证:

$\begin{bmatrix} x+T_x \\ y+T_y \\ z+T_z \\ 1 \end{bmatrix} = \begin{bmatrix}
    1 & 0 & 0 & T_x \\ 
    0 & 1 & 0 & T_y \\ 
    0 & 0 & 1 & T_z \\ 
    0 & 0 & 0 & 1 
    \end{bmatrix} \begin{bmatrix} x \\ y \\ z \\ 1 \end{bmatrix}$

\vspace{3em}

$A = \begin{bmatrix} 1 & 0 & 0 & -E_x \\
            0 & 1 & 0 & -E_y \\
            0 & 0 & 1 & -E_z \\
            0   &   0 &    0 & 1 
\end{bmatrix}$

\vspace{3em}


$B = \begin{bmatrix} S_x & S_y & S_z & 0 \\
                     U_x & U_y & U_z & 0 \\
                     -F_x & -F_y & -F_z & 0 \\
                     0   &   0 &    0 & 1 

 \end{bmatrix}$

 \vspace {3em}

 将这两个变换组合到一起就成了$V$变换矩阵。先进行变换$A$,
 再进行变换$B$,则$V=BA$。(因为向量在矩阵右边与矩阵相乘,所以先进行的变换后乘。例如向量$\vec p$,则由于矩阵乘法满足结合律,有
 $BA\vec p = B(A\vec p)$,先进行了变换$A$,再对变换后的$A\vec p$进行$B$的变换)

$V = BA = \begin{bmatrix} S_x & S_y & S_z & 0 \\
    U_x & U_y & U_z & 0 \\
    -F_x & -F_y & -F_z & 0 \\
    0   &   0 &    0 & 1 

\end{bmatrix} \begin{bmatrix} 1 & 0 & 0 & -E_x \\
    0 & 1 & 0 & -E_y \\
    0 & 0 & 1 & -E_z \\
    0   &   0 &    0 & 1 
\end{bmatrix}
= \begin{bmatrix} S_x & S_y & S_z & -\vec S \cdot \vec E \\
                          U_x & U_y & U_z & -\vec U \cdot \vec E \\
                          -F_x & -F_y & -F_z & \vec F \cdot \vec E \\
                          0 & 0 & 0 & 1
\end{bmatrix}$

\newpage 

设投影前坐标范围的最大值和最小值分别为$(X_{max}, Y_{max}, Z_{far})$和$(X_{min}, Y_{min}, Z_{near})$,投影矩阵为

$P = \begin{bmatrix} 
\frac{2}{X_{max}-X_{min}} & 0 & 0 & -\frac{X_{max}+X_{min}}{X_{max}-X_{min}}\\
0 & \frac{2}{Y_{max}-Y_{min}} & 0 & -\frac{Y_{max}+Y_{min}}{Y_{max}-Y_{min}}\\
0 & 0 & \frac{2}{Z_{far}-Z_{near}} & -\frac{Z_{far}+Z_{near}}{Z_{far}-Z_{near}}\\
0 & 0 & 0 & 1\\ \end{bmatrix}$

\newpage 

在上图中我们设Projection Window的半高为1(高为2),半宽则为宽高比$r$(宽为$2r$)
由几何关系,我们有从眼睛到Projection Window的距离$d$满足$\tan\frac{\alpha}{2} = \frac{1}{d}$,从而$d = \cot\frac{\alpha}{2}$。
%而上图中的$\beta$角满足$\tan\frac{\beta}{2} = \frac{r}{d} = r \tan\frac{\alpha}{2}$

透视投影中,我们希望将一个坐标$(x,y,z)$变换到Projection Window上,也就是坐标变换为$(x', z', -d)$(Projection Window在平面$z=-d$上)
由相似三角形可得:

\begin{align}
\frac{x'}{-d} &= \frac{x}{z} \\
x' = \frac{x(-d)}{z} &= -\frac{x\cot\frac{\alpha}{2}}{z}
\end{align}

同理有

\begin{align}
y' &= -\frac{y\cot\frac{\alpha}{2}}{z}
\end{align}

投影后的$x'$, $y'$若在Projection Widnow上,则满足:
\begin{align}
 -r& \le x' \le r \\
 -1&\le y' \le 1 \\
 n & \le z \le f 
\end{align}

对于变换后的$x$坐标,我们也希望把它归一化到$[-1,1]$区间内,则再令$x'$除以$r$,于是最终我们得到:

\begin{align}
    x' &= -\frac{x\cot\frac{\alpha}{2}}{rz} \\
    y' &= -\frac{y\cot\frac{\alpha}{2}}{z} \\
    -1& \le x' \le 1 \\
    -1&\le y' \le 1 \\
    n & \le z \le f 
\end{align}

\newpage

其实到这里$x,y$的变换就已经算是完成了,但是还有很大的问题就是$x', y'$的表达式中含有透视投影的参数($\alpha,n,f,r$)之外的$z$。接下来我们想办法解决这个问题,并完成$z$坐标的投影。我们想到透视投影之后,OpenGL会进行一个透视除法将点$(x,y,z,w)$的四个分量都除以$w$分量,
而$x', y'$的表达式也都只是在分母上有一个$z$,因此我们可以想办法让透视投影变换后的$w$分量等于$z$或者$-z$,而$x, y$的变换与$z$无关,于是我们构造一个这样的投影矩阵$P$,与$z$无关:

\vspace{1em}

$ P = \begin{bmatrix} \frac{\cot\frac{\alpha}{2}}{r} & 0 & 0 & 0 \\
                      0 & \cot\frac{\alpha}{2} & 0 & 0 \\
                      0 & 0 & A & B \\
                      0 & 0 & -1 & 0 
    \end {bmatrix}$


矩阵对$x, y$的变换是已经确定了的,而对$z$的变换还不确定,因此我们设两个未知数$A$,$B$并在之后求解他们,先求得变换后的坐标:

\vspace{1em}

$\begin{bmatrix} \frac{\cot\frac{\alpha}{2}}{r} & 0 & 0 & 0 \\
    0 & \cot\frac{\alpha}{2} & 0 & 0 \\
    0 & 0 & A & B \\
    0 & 0 & -1 & 0 
\end {bmatrix} \begin{bmatrix} x \\ y \\ z \\ 1 \end{bmatrix} 
= \begin{bmatrix} \frac{x\cot\frac{\alpha}{2}}{r} \\ {y\cot\frac{\alpha}{2}} \\ Az+B \\ -z\end{bmatrix}
$

之后进行透视除法,也就是所有分量除以$w$分量也就是$-z$得到
\vspace{1em}


$ \begin{bmatrix} -\frac{x\cot\frac{\alpha}{2}}{rz} \\ -\frac{y\cot\frac{\alpha}{2}}{z} \\ -(A+\frac{B}{z}) \\ 1\end{bmatrix} $

可以看到,前两个分量的表达式与前面求得的$x',y'$是一致的。而对于$z$分量,
我们可以看到投影变换并且进行透视除法之后的$z$
坐标应该变为$-(A+\frac{B}{z})$。
而且透视除法后进入NDC空间改用左手坐标系,那么近平面上$z = -n$时,应该对应NDC空间的$z$坐标$-1$,则$-(A+\frac{B}{z}) = -1$,
而远平面上$z = -f$对应NDC空间$z=1$,则有$-(A+\frac{B}{z}) = 1$,于是得到方程组:

\begin{align}
-(A+\frac{B}{-n}) &= -1 \\
-(A+\frac{B}{-f}) &= 1 
\end{align}

从而解得
\begin{align}
A &= -\frac{f+n}{f-n}\\
B &= -\frac{2nf}{f-n} 
\end{align}

\newpage

于是最终有投影矩阵$P$为:

\vspace{1em}

$ P = \begin{bmatrix} \frac{\cot\frac{\alpha}{2}}{r} & 0 & 0 & 0 \\
    0 & \cot\frac{\alpha}{2} & 0 & 0 \\
    0 & 0 & -\frac{f+n}{f-n} & -\frac{2nf}{f-n}  \\
    0 & 0 & -1 & 0 
\end {bmatrix}$

\end{document}
