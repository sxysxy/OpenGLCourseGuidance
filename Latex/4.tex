%!TEX program=xelatex
\documentclass[UTF8]{ctexart}
\usepackage{amsmath}        
\usepackage{geometry}
\geometry{left=2.5cm, right=2.5cm, top=2.5cm, bottom=2.5cm}  %重设页边距(默认的页边距太大,不美观)
\usepackage{xeCJK}      %字体包
%\usepackage{fontspec}
\begin{document}

旋转变换则要复杂许多,一个比较好的旋转模型是,绕一个给定的轴$\vec n$,
围绕$\vec n$旋转$\theta$角。下面给出旋转变换的原理和公式推导方法:
假设给出的转轴$\vec n$长度为1(即$\vec n$是转轴方向上的单位向量),
要旋转的向量(点也成立,这里以向量为例)为$\vec v$,
完成旋转后的向量为$R_n(\vec v)$。

首先将$\vec v$分解为两部分,一部分平行于
$\vec n$而另一部分垂直于
$\vec n$,
平行于$\vec n$的分量即为
$\vec v$在$\vec n$上的投影
$Proj_n(\vec v)$,
则另一分量可以得出$\vec{v_{\perp}} = \vec{v} - Proj_n(\vec v)$,
接下来求得在图中与轴$\vec n$垂直的圆平面上,$\vec{v_{\perp}}$旋转$\theta$后的$R_n(\vec{v_{\perp}})$,为此我们在该
圆平面上找两个正交的向量作为基,一个向量就是$\vec{v_{\perp}}$,而另一个向量则取$\vec n \times \vec v$($\times$是向量叉乘):

这样就可以求得$R_n(\vec{v_{\perp}}) = \vec{v_{\perp}} \cos\theta + (\vec n \times \vec v) \sin\theta$,
那么最后要求的$R_n(\vec v) = R_n(\vec{v_{\perp}}) + Proj_n(\vec v)$,展开并做变形:

\begin{align}
R_n(\vec v)  &= R_n(\vec{v_{\perp}}) + Proj_n(\vec v) \\
    &= \vec{v_{\perp}}\cos\theta + (\vec n \times \vec v) \sin\theta + (\vec n \cdot \vec v)\vec n \\
    &= (\vec n \cdot \vec v)\vec n + (\vec v - (\vec n \cdot \vec v)\vec n)\cos\theta + (\vec n \times \vec v)\sin\theta \\
    &= \vec{v} \cos\theta + (\vec n \cdot \vec v)\vec{n}(1-\cos\theta) + (\vec n \times \vec v)\sin \theta 
\end{align}

依照上面的结果给出绕轴$n$旋转$\theta$角度的变换矩阵$R$,要求$\vec n$是单位向量,式中$x$, $y$, $z$分别是$\vec n$的三个分量,$c = \cos\theta$, $s = \sin\theta$

$R = \begin{bmatrix}
    c+(1-c)x^2 & (1-c)xy+sz & (1-c)xz-sy & 0 \\ 
    (1-c)xy-sz & c+(1-c)y^2 & (1-c)yz+sx & 0\\ 
    (1-c)xz+sy & (1-c)yz-sx & c+(1-c)z^2 & 0\\ 
    0 & 0 & 0 & 1 
    \end{bmatrix} $

值得注意的是,旋转变换的逆变换即反方向旋转$\theta$的变换,将$\theta$代换为$-\theta$后可以得到逆变换矩阵$R^{-1}$,它恰好等于$R$的转置$R^T$。即旋转变换矩阵$R$是一个正交矩阵。

\newpage

而平移变换就简单多了,我们想将坐标$(x, y, z, 1)$变换到$(x+T_x, y+T_y, z+T_z, 1)$,容易验证:

$\begin{bmatrix} x+T_x \\ y+T_y \\ z+T_z \\ 1 \end{bmatrix} = \begin{bmatrix}
    1 & 0 & 0 & T_x \\ 
    0 & 1 & 0 & T_y \\ 
    0 & 0 & 1 & T_z \\ 
    0 & 0 & 0 & 1 
    \end{bmatrix} \begin{bmatrix} x \\ y \\ z \\ 1 \end{bmatrix}$

\vspace{3em}

$A = \begin{bmatrix} 1 & 0 & 0 & -E_x \\
            0 & 1 & 0 & -E_y \\
            0 & 0 & 1 & -E_z \\
            0   &   0 &    0 & 1 
\end{bmatrix}$

\vspace{3em}


$B = \begin{bmatrix} S_x & S_y & S_z & 0 \\
                     U_x & U_y & U_z & 0 \\
                     -F_x & -F_y & -F_z & 0 \\
                     0   &   0 &    0 & 1 

 \end{bmatrix}$

 \vspace {3em}

 将这两个变换组合到一起就成了$V$变换矩阵。先进行变换$A$,
 再进行变换$B$,则$V=BA$。(因为向量在矩阵右边与矩阵相乘,所以先进行的变换后乘。例如向量$\vec p$,则由于矩阵乘法满足结合律,有
 $BA\vec p = B(A\vec p)$,先进行了变换$A$,再对变换后的$A\vec p$进行$B$的变换)

$V = BA = \begin{bmatrix} S_x & S_y & S_z & 0 \\
    U_x & U_y & U_z & 0 \\
    -F_x & -F_y & -F_z & 0 \\
    0   &   0 &    0 & 1 

\end{bmatrix} \begin{bmatrix} 1 & 0 & 0 & -E_x \\
    0 & 1 & 0 & -E_y \\
    0 & 0 & 1 & -E_z \\
    0   &   0 &    0 & 1 
\end{bmatrix}
= \begin{bmatrix} S_x & S_y & S_z & -\vec S \cdot \vec E \\
                          U_x & U_y & U_z & -\vec U \cdot \vec E \\
                          -F_x & -F_y & -F_z & \vec F \cdot \vec E \\
                          0 & 0 & 0 & 1
\end{bmatrix}$

\end{document}
